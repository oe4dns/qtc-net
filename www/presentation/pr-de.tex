\documentclass{seminar}
\usepackage[display]{texpower}
\usepackage[utf8]{inputenc}
\usepackage{german}
\usepackage{graphicx}
\usepackage{longtable}
\usepackage{hyperref}

%\usepackage[usenames,dvipsnames]{xcolor}
%\definecolor{ueblue}{rgb}{0,0,0.2}
%\definecolor{ueblue}{rgb}{0.6,0.6,1}
%\everymath{\color{yellow}}
%\everydisplay{\color{yellow}}
%\pagecolor{ueblue}
%\pagecolor{ueblue}
%\color{white}

% comment out to have info centered on slides
\centerslidesfalse
\providecommand{\T}[1]{
	\begin{center}
		{\bf #1}
	\end{center}
	\vspace{2mm}
	\hrule
	\vspace{2mm}
}

\newpagestyle{mypagestyle}%
{}%
{https://www.qtc-net.org \hfil \sl QTC-Net Messaging}


% maketitle
\title{QTC-Net Short Telegram Service}
\author{OE1SRC Hans Freitag }

\begin{document}

\pagestyle{empty}
\centerslidestrue
\begin{slide}
	\maketitle
	\begin{center}
		QTC Net, ein Kurztelegramm- und Microblog- system für Amateurfunk
		{\sf (Rechtschreibfehler sind ein Feature!)}
	\end{center}
\end{slide}
\centerslidesfalse

\pagestyle{mypagestyle}

\begin{slide}
	\T{Die Idee\ldots}
	Amateurfunk bietet abseits vom Sprechfunk viele fantastische Möglichkeiten zu kommunizieren, mittlerweile sind auch einige Kurzmitteilungssysteme dabei. QTC Net ist eine Infrastruktur um alle diese Systeme zu vernetzen. \\
 \pause
Die Idee selbst ist nicht neu.... 
\end{slide}

\begin{slide}
	\T{APRS}
	Bob Bruniga (WB4APR) hat vor Jahren einmal vorgeschlagen den APRS Backbine zu verwenden um kurzmitteilungssysteme miteinander zu vernetzen. \\ \pause
	\begin{itemize}
		\item dezentral, 
		\item begrenzte Nachrichtenlänge, 
		\item begrenzte Rufzeichenvarianten, 
		\item echtzeitmeldesystem, sender und empfänger müssen gleichzeitig aktiv sein
		\item passwörter werden aus dem Rufzeichen generiert (security)
	\end{itemize}
\end{slide}

\begin{slide}
	\T{Welche Infrastruktur?}
	\begin{itemize}
		\item Dezentrale Infrastruktur mit Ortsinformationen? wie E-Mail, oder BBS systeme.
		\item Zentrale Infrastruktur mit einem Server auf den alle zugreifen?
		\item Selbstorganisierende dezentrale Infrastruktur ohne Ortsinformationen.
	\end{itemize}
\end{slide}

\begin{slide}
	\T{Geht eine Selbstorganisierende Infrastruktur überhaupt?}
	Mal sehen was wir wissen: 
	\begin{itemize}
		\item Anzahl Funkamateure Weltweit 2,8 Millionen
		\item Größe einer QTC-Net Nachricht 400-700 Byte. 
		\item Es gibt eine Auslieferungsbestätigung 360 Byte
		\item Annahme, jeder verschickt 2 Nachrichten pro Tag. 
	\end{itemize}
\end{slide}

\begin{slide}
	\T{Geht eine Selbstorganisierende Infrastruktur überhaupt?}
	Etwas Mathe: 
	$$
		2.8\cdot10^6 \cdot 2 \cdot (700B+360B) = 5936 \cdot10^6 B \approx 5.5 GB 
	$$
\end{slide}

\begin{slide}
	\T{Haben wir einen Backbone der 5.5 GB Tagesvolumen übersteht?}
	HamNet Vielleicht? 22 MBit/s. 
	$$ (22MBit/s\cdot1024\cdot1024)/8=2883584 Byte/s $$
	$$ 2883584 Byte/s \cdot 86400s \approx 232 GB$$
	HamNet hat ausreichend reserve. \\
	Schmalbandigere anbindungen sind in der Lage die für sie 
	wichtigen Teile des Datenbestandes zu syncronisieren.   
\end{slide}

\begin{slide}
	\T{Authentifizierung mit elektronischer Signatur}
	Eine Amateurfunkaussendung ist für die Öffentlichkeit 
   bestimmt, das ist gut so, stellt aber bei der 
	Identitätsprüfung ein problem dar. 

	Signaturen lösen dieses Problem haben den Vorteil, das 
	kein Passwort über Radiolinks übertragen werden muss.
\end{slide}

\begin{slide}
	\T{Nachrichtentypen}
	\begin{itemize}
		\item telegram
		\item qsp empfangsbestätigung
		\item public key
		\item key revoke 
		\item operator Identität
		\item trust 
	\end{itemize}
\end{slide}

\end{document}
