\documentclass{article}
\usepackage{hyperref}


\title{QTC-Net Innovations Price Applicaion}
\begin{document}
\maketitle


\begin{abstract}
What if HAM A wants to send a Message to HAM B but he does not have any 
Phone Number, Address, information where his friend is and what activities 
he is involved in? He will send a QTC-NET Telegram in the future. 
\end{abstract}

\section{Goals}

The Goal of this project is to create something that:

\begin{itemize}
	\item should just use the callsigns for addressing, no routing info 
	\item should know when somebody owns more than one callsign
	\item is widely accepted, relieable
	\item should not fail in case of a catastrophic event
	\item can legally distributed over ham radio links
	\item provides more data-safety than a late 80th solution
\end{itemize}

This is a huge task. To be widely accepted the new thing must fit into almost 
any existing infrastructures like HamNet, Internet, Packet Radio, DXCluster, 
QSL Databases, DStar, DMR, APRS, and even Manual Operations, and systems I 
forgot about. For example, if someone types in a telegram in crzcq someone 
the receiver who may have his callbook entry on QRZ.com, can read it there. 
I have choosen these two projects as an example because they have their 
differences, and it illustrates perfectly that you to not go with a simple 
database. \\

If you do someone opens the next database because he does not feel 
comfortable with your work. However this may happen anyway, qtc-net can deal 
with it. \\

All this has lead me to the design of QTC-Net.  


\section{Architecture}

The main component of qtc net is a public key signed message, which is
distributed between all nodes\footnote{It except for the bockchain data, 
it is possible to filter down the amount of messages until the amount 
of traffic is low enough for your line}.\\

A message is one of the following types: 

\begin{description}
	\item[telegram] a standard text message
	\item[qsp] states if a telegram was received
	\item[operator] holds operator Information
	\item[key] holds an operators public key
	\item[revoke] points out if an operators key is revoked
	\item[trust] publishes a trustlevel for a public key
\end{description}


Every Message is signed with a public key owned by the creators callsign. 
Every Message is public, too. So everyone can read it. Because of the keys 
and trustlevels no extra authentication is needed if you are a node. \\

\subsection{Roles and Aliases}

A telegram contains 3 callsigns, the sender of the telegram (from), he 
receiver of the telegram (to) and the creator of the message (call). This 
makes it possible that the creator of a message picks up a telegram from any 
sender to any receiver, so that sender and receiver do not need to have a 
public key stored in the network.\\

The Advantage of those 3 role concept is that even RTTY, PSK31 and CW links 
can be used together with QTC Net Telegrams, on APRS for example, a message 
is send by the sender to a receiver, but signed and placed into QTC Net by 
one of the APRS Gateways callsigns.\\ 

The QSP (deliver without fee) Message has two callsigns, the receiver of 
the telegram that is delivered (to), and again the messages creator (call). It
may happen that a telegram to oe1src is delivered to dd5tt which is the same OP
but who just owns two callsigns. The qsp-to keeps track of this. \\

The operator message contains a set-of-aliases for the creator of the aliases 
message (call). For example oe1src and dd5tt but also oe1src-7 and 9a/oe1src 
are aliases of oe1src. This also means that if there is a resource oe1src-7 
on aprs it can receive and tell the gateway to qsp (mark as delivered) a qtc-net 
telegram for oe1src. \\

An operator can also follow several different other cals with the set-of-followings
list in the operator message.  If you follow a call, you get every message to that call 
listed  too when you query for messages to your call, but not as new messages, those 
messages are getting sorted into a timeline and timeline-new message list\\

An Operator can have more than one different public keys, one for each 
node he is working with. He needs to sign both of them against each other. \\

\subsection{Syncronization}

The Server Nodes can syncronize each other by using different protocols, right
now there only is an http based sync protocol defined and implemented, but this
will change in the near future. The current http based protocol may be used to 
sync the nodes over HamNet or the Internet. A Syncronization over other radio 
links is also planned.  \\

\section{Development}

QTC Net is developed in Perl for Unix.

The Project is released as GPLv3 and the development takes place on GIT-HUB,
this means that everybody can help with the development, and has easy access
even to old revisions. The Project has a good reference Documentation about
every part of the solution to make it easy to understand how it works. \\

This Open Developent is not common some Ham Radio Software, but it should so 
because Ham Radio is an Open Radio Service. \\

\section{3rd Party Software}

QTC Net depends on 3rd party Ham Radio Systems utilizing it for telegram
transport. Currently there is a Web Frontent delivered with the QTC-Net Node
Software, which can be installed on your webserver. This is the same Web
frontend that powers \url{https://www.qtc-net.org/} \\

A second frontend is the APRS Gateway which behaves like an APRS IGATE. It
translates QTC Messages between APRS and QTC-Net. A nice side effect of the
gate is, that APRS messages are now stored in QTC-Net if the receiver does 
not answer in time, and is forwarded as soon as the receiving station 
appears. \\

A third way to access is the DXSpider DXCluster software. There is a funktion 
available which shows how many new QTC messages a spotted call has. So you may
tell him the messages when you have a QSO with the spotted OP. The DXSpider
Plugin can also send, show and qsp telegrams in QTC-Net. \\

There is an RSS News Feed Generator for QTC Net. This Piece of software will 
show you new or timelined QTC Telegrams as RSS Feeds in your newsfeed reader.
Theese Feeds can also be forwarded to Twitter and other Microblogging sites if
needed. \\ 

Planned Plugins go for a QTC Net enabled PSKMail system together with some 
PSK31 and CW Bots for shortwave radio access. 


\section{Criteria}

\subsection{Age}

The Project is only afew month older than one year due to the fact
that there was no Innovation price last year. However it is still a young 
project. I hope that the Innovations price is defines the project still as 
new. 

\subsection{Public Relations}

You can see the Projects Website and the Development in the Internet. The 
code is released unter GPLv3 and can be used by everyone. Everyone can help 
with the development by submitting Issues on Git-Hub or by Integrating the 
Projects data into his website. By showing that there is new developent in Ham 
Radio the Project makes our radio service interesting to others.


\subsection{Youth empowerment}

New technologies play a key role for Youth empowerment, they show that Ham Radio 
isn't only an old mans hobby and they are making ham radio atractive to young people 
who have an interest to experiment with new technical innovations. During Talks on 
Hacker Congresses and other Social Media events I was able to attract many young 
people for Ham Radio and it applications. 

\end{document}
